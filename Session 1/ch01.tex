% Created 2018-11-20 Tue 23:17
% Intended LaTeX compiler: pdflatex
\documentclass[11pt]{article}
\usepackage[utf8]{inputenc}
\usepackage[T1]{fontenc}
\usepackage{graphicx}
\usepackage{grffile}
\usepackage{longtable}
\usepackage{wrapfig}
\usepackage{rotating}
\usepackage[normalem]{ulem}
\usepackage{amsmath}
\usepackage{textcomp}
\usepackage{amssymb}
\usepackage{capt-of}
\usepackage{hyperref}
\author{The NEXUS, SJCET}
\date{\today}
\title{Basic HTML Tags}
\hypersetup{
 pdfauthor={The NEXUS, SJCET},
 pdftitle={Basic HTML Tags},
 pdfkeywords={},
 pdfsubject={},
 pdfcreator={Emacs 26.1 (Org mode 9.1.9)}, 
 pdflang={English}}
\begin{document}

\maketitle
\section*{Basic HTML Structure}
\label{sec:org85f209f}
\begin{verbatim}
<!DOCTYPE html>
<html>
<head>
<title>website title</title>
</head>
<meta charset="utf-8">
<body>
content of website ...
</body>
</html>
\end{verbatim}

\begin{verbatim}
<!DOCTYPE html>
\end{verbatim}

This is what every website should start with. This line basically tells your browser (the thing you use to browse the web) that the website is using the latest version of HTML, HTML5.

\begin{verbatim}
<html>
\end{verbatim}

This is the starting html tag.

You now may be wondering what 'tags' are. In their simplest form, tags tell the browser what to do with the stuff inside of them.

If you take a took at the last line of code, you'll see the ending html tag. You can tell it's the ending tag because it has a forward slash before the name of the tag.

Everything between the starting and ending tags is affected by them. In this case, html tags tell the browser that whatever is inside of them should be part of the website.

\begin{verbatim}
<head>
\end{verbatim}

More tags! Yes, you can have tags inside of tags.

The head tag can contain a lot of stuff like styles and scripts, but in this example it contains two things. Let's take a look at them! Sidenote: you can't see anything in the head tags.

\begin{verbatim}
<meta charset="utf-8">
\end{verbatim}

All this scary looking line does is tell the browser which character set you want to use. Basically, by adding this line, you get more characters. Hooray! Sidenote: meta tags don't need to be ended.

\begin{verbatim}
<title>Project2 Tutorials</title>
\end{verbatim}

Yes, you can have the starting and ending tags on the same line, which is very useful if you don't have much stuff in them, like here!

The title tag tells the browser what to put in the tabs at the top, or the title of the browser window. In this case, it would put Project2 Tutorials in the tab.

\begin{verbatim}
<body>
\end{verbatim}

The body 
tag contains all the stuff that you can see on the website, such as text, images, and links. In this example, we only have one thing inside the body tags.

\begin{verbatim}
<h1>Welcome!</h1>
\end{verbatim}

The h1 tag makes all the text inside of it big and bold! In this case, the word 'welcome' is that text.

\section*{Common Tags}
\label{sec:orge685565}
\begin{verbatim}
-  <h?> heading </h?>	                Heading (h1 for largest to h6 for smallest)
-  <p> paragraph </p>	                Paragraph of Text
-  <b> bold </b>	                Make text between tags bold
-  <i> italic </i>	                Make text between tags italic
-  <a href="url"> link name </a>        Create a link to another page or website
-  <div> ... </div>	                Divide up page content into sections, and applying styles
-  <img src="filename.jpg"> 	        Show an image
-  <ul> <li> list </li> </ul>	        Unordered, bullet-point list
-  <br> 	                        Line Break (force a new line)
-  <span style="color:red"> red </span> Use CSS style to change text colour
\end{verbatim}
\section*{Text Formatting}
\label{sec:orgcabbc05}
\begin{verbatim}
<h?> ... </h?>                        Heading (?= 1 for largest to 6 for smallest, eg h1)
<b> ... </b>                          Bold Text
<i> ... </i>                          Italic Text
<u> ... </u>                          Underline Text
<strike> ... </strike>                Strikeout
<sup> ... </sup>                      Superscript - Smaller text placed below normal text
<sub> ... </sub>                      Subscript - Smaller text placed below normal text
<small> ... </small>                  Small - Fineprint size text
<tt> ... </tt>                        Typewriter Text
<pre> ... </pre>                      Pre-formatted Text
<blockquote> ... </blockquote>        Text Block Quote
<strong> ... </strong>                Strong - Shown as Bold in most browsers
<em> ... </em>                        Emphasis - Shown as Italics in most browsers
<font> ... </font>                    Font tag obsolete, use CSS. (*)
\end{verbatim}
\section*{Section Divisions}
\label{sec:orgfd415b6}
\begin{verbatim}
<div> ... </div>              	Division or Section of Page Content
<span> ... </span>            	Section of text within other content
<p> ... </p>                          Paragraph of Text
<br>                                  Line Break
<hr>                          	Basic Horizontal Line
<hr>
\end{verbatim}
\subsection*{Tag Attributes:}
\label{sec:org72c6b91}
\begin{verbatim}
size="?"	                        Line Thickness in pixels
width="?"                           	Line Width in pixels
width="??%"                        	Line Width as a percentage
color="#??????"               	Line Colour (*)
align="?"                    	Horizontal Alignment: left, center, right (*)
noshade                      	No 3D cut-out
<nobr> ... </nobr>           	Line Break

\end{verbatim}
\section*{Images}
\label{sec:orgeabbfb8}
\begin{verbatim}
<img src="url" alt="text"> 	        Basic Image
<img>                                 Tag Attributes:	 
src="url"	                        URL or filename of image (required!)
alt="text"	                        Alternate Text (required!)
align="?"	                        Image alignment within surrounding text (*)
width="??"	                        Image width (in pixels or %)
height="??"	                        Image height (in pixels or %)
border="??"	                        Border thickness (in pixels) (*)
vspace="??"	                        Space above and below image (in pixels) (*)
hspace="??"	                        Space on either side of image (in pixels) (*)
\end{verbatim}
\section*{Linking Tags}
\label{sec:org6681346}
\begin{verbatim}
<a href="url"> link text </a>	        Basic Link
<a>                                   Tag Attributes:	 
href="url"		                Location (url) of page to link to.
name="??"		                Name of link (name of anchor, or name of bookmark)
target="?"		                Link target location: _self, _blank, _top, _parent.
href="url#bookmark"		        Link to a bookmark (defined with name attribute).
href="mailto:email"		        Link which initiates an email (dependant on user's email client).
\end{verbatim}
\section*{Lists}
\label{sec:orga93986d}
\begin{verbatim}
<ol> ... </ol>	                Ordered List
<ul> ... </ul>	                Un-ordered List
<li> ... </li>	                List Item (within ordered or unordered)
<ol type="?">		                Ordered list type: A, a, I, i, 1
<ol start="??">	                Ordered list starting value
<ul type="?">		                Unordered list bullet type: disc, circle, square
<li value="??">	                List Item Value (changes current and subsequent items)
<li type="??">	                List Item Type (changes only current item)
<dl> ... </dl>	                Definition List
<dt> ... </dt>	                Term or phrase being defined
<dd> ... </dd>	                Detailed Definition of term
\end{verbatim}
\section*{Tables}
\label{sec:org1fbd192}
\begin{verbatim}
<table> ... </table>                  Define a Table
<tr> ... </tr>	                Table Row within table
<th> ... </th>	                Header Cell within table row
<td> ... </td>	                Table Cell within table row
\end{verbatim}
\subsection*{<table> Tag Attributes:}
\label{sec:org8ea0368}
\begin{verbatim}
border="?"	                        Thickness of outside border
bordercolor="#??????"                 Border Colour
cellspacing="?"                       Space between cells (pixels)
cellpadding="?"                       Space between cell wall and content
align="??"	                        Horizontal Alignment: left, center, right (*)
bgcolor="#??????"                     Background Colour (*)
width="??"	                        Table Width (pixels or %) (*)
height="??"	                        Table Height (pixels or %) (*)
\end{verbatim}
\subsection*{<td> Tag Attributes:}
\label{sec:orge19ff92}
\begin{verbatim}
colspan="?"		        Number of columns the cell spans across (cell merge)
rowspan="?"		        Number of row a cell spans across (cell merge)
width="??"	        	        Cell Width (pixels or %) (*)
height="??"		        Cell Height (pixels or %) (*)
bgcolor="#??????"  	        Background Colour (*)
align="??"	        	        Horizontal Alignment: left, center, right (*)
valign="??"		        Vertical Alignment: top, middle, bottom (*)
nowrap		                Force no line breaks in a particular cell
\end{verbatim}
\section*{Frames}
\label{sec:org35b7423}
\begin{verbatim}
<frameset> ... </frameset>	        Define the set of Frames
<frame> ... </frame>	                Define a frame within the frameset
<noframes> ... </noframes>	        Unframed content (for browsers not supporting frames)
\end{verbatim}
\subsection*{<frameset> Tag Attributes:}
\label{sec:org6444e01}
\begin{verbatim}
rows="??,??, ..."		        Define row sizes & number of rows (size in pixels or %)
cols="??,??, ..."		        Define column sizes & number of columns (size in pixels or %)
noresize="noresize"		        User cannot resize any frames in frameset
\end{verbatim}
\subsection*{<frame> Tag Attributes:}
\label{sec:org4c43c2b}
\begin{verbatim}
src="url"		                Location of HTML File for a frame
name="***"		                Unique name of frame window
marginwidth="?"		        Horizontal margin spacing inside frame (pixels)
marginheight="?"		        Vertical margin spacing inside frame (pixels)
noresize="noresize"		        Declare all frameset sizes as fixed
scrolling="***"		        Can the user scroll inside the frame: yes, no, auto
frameborder="?"		        Frame Border: (1=yes, 2=no)
bordercolor="#??????"	        Border Colour (*)
\end{verbatim}
\section*{Forms}
\label{sec:orgc1e388c}
\begin{verbatim}
<form> ... </form>	                Form input group decleration
<input> ... </input>	                Input field within form
<select> ... </select>	        Select options from drop down list
<option> ... </option>	        Option (item) within drop down list
<textarea> ... </textarea>	        Large area for text input
\end{verbatim}

\subsection*{<form> Tag Attributes:}
\label{sec:org6416c21}
\begin{verbatim}
action="url"		                URL of Form Script
method="***"		                Method of Form: get, post
enctype="***"	                For File Upload: enctype="multipart/form-data"
\end{verbatim}
\subsection*{<input> Tag Attributes:}
\label{sec:org1264b93}
\begin{verbatim}
type="***"		                Input Field Type: text, password, checkbox, submit etc.
name="***"		                Form Field Name (for form processing script)
value="***"		                Value of Input Field
size="***"		                Field Size
maxlength="?"	                Maximum Length of Input Field Data
checked		                Mark selected field in radio button group or checkbox
\end{verbatim}

\subsection*{<select> Tag Attributes:}
\label{sec:orgd341161}
\begin{verbatim}
name="***"		                Drop Down Combo-Box Name (for form processing script)
size="?"		                Number of selectable options
multiple		                Allow multiple selections
\end{verbatim}
\subsection*{<option> Tag Attributes:}
\label{sec:org1728f98}
\begin{verbatim}
value="***"		                Option Value
selected		                Set option as default selected option
\end{verbatim}
\subsection*{<textarea> Tag Attributes:}
\label{sec:org0eaf34d}
\begin{verbatim}
name="***"		                Text Area Name (for form processing script)
rows="?"		                Number of rows of text shown
cols="?"		                Number of columns (characters per rows)
wrap="***"		                Word Wrapping: off, hard, soft
\end{verbatim}
\section*{Special Characters}
\label{sec:orga345037}
\begin{verbatim}
&lt;	 <    - Less-Than Symbol
&gt;	 >    - Greater-Than Symbol
&amp;	 &    - Ampersand, or 'and' sign
&quot; "    - Quotation Mark
&copy; ©    - Copyright Symbol
&trade;™    - Trademark Symbol
&nbsp; -    A space (non-breaking space)
\end{verbatim}

\section*{Miscellaneous Tags}
\label{sec:orge623109}
\begin{verbatim}
<!-- ... -->	                         Comment within HTML source code
<!DOCTYPE html ... >	                 Document Type Definition (wiki)
<meta> ... </meta>	                 META information tag
<meta>                                 Tag Attributes:	 
name="***"	                         Meta name: description, keywords, author
http-equiv="***"	                 HTTP Equivalent Info: title, etc.
content="***"	                         Information content
<link>	                         LINK content relationship tag
<link>                                 Tag Attributes:	 
rel="***"	                         Type of forward relationship
http="url"	                         Location (URL) of object or file being linked
type="***"	                         Type of object or file, eg: text/css
title="***"	                         Link title (optional)

\end{verbatim}
\section*{Body Background \& Colours}
\label{sec:orgba8bb6f}
\begin{itemize}
\item <body> Tag Attributes:	 
\begin{verbatim}
background="url"	                 Background Image (*)
bgcolor="#??????" 	                 Background Colour (*)
text="#??????" 	                 Document Text Colour (*)
link="#??????" 	                 Link Colour (*)
vlink="#??????" 	                 Visited Link Colour (*)
alink="#??????" 	                 Active Link Colour (*)
bgproperties="fixed"                  Background Properties - "Fixed" = non-scrolling watermark (*)
leftmargin="?" 	                 Side Margin Size in Pixels (Internet Explorer) (*)
topmargin="?" 	                 Top Margin Size in Pixels (Internet Explorer) (*)
\end{verbatim}
\end{itemize}
\end{document}
